\section{Previous concepts:}

\subsection{Image-Language models(IMLs):}
\noindent An image-language model is an artificial intelligence system designed to process and understand both visual and textual data, integrating these two modalities to perform various tasks. These models can generate text descriptions from images, produce images from text descriptions, and understand the relationships between visual and textual elements. Key capabilities and applications of image-language models include:

\begin{itemize}
\item \textbf{Image Captioning:} Generating descriptive text for a given image.
\item \textbf{Visual Question Answering (VQA):} Answering questions related to the content of an image.
\item \textbf{Image Generation from Text:} Creating images based on textual descriptions.
\item \textbf{Cross-modal Retrieval:} Finding images that match a given text or vice versa.
\item \textbf{Image-Text Matching:} Evaluating the relevance or similarity between an image and a text description.
\end{itemize}

\noindent These models typically combine techniques from computer vision (for processing videos) and natural language processing (for handling text). They often use deep learning architectures such as:

\begin{itemize}
\item \textbf{Convolutional Neural Networks (CNNs):} For extracting features from images.
\item \textbf{Transformers:} For handling and generating text, and sometimes for processing image features.
\item \textbf{Multimodal Models:} Like CLIP (Contrastive Language–Image Pretraining) and DALL-E, developed by OpenAI, which are specifically designed to understand and generate both images and text.
\end{itemize}
\subsection{Video-Language models(VidLMs):}
\noindent Video-language models are advanced machine learning systems designed to understand and generate both video content and associated natural language descriptions. These models can interpret and generate content involving the complex interplay between visual data (videos) and textual data (language). Key capabilities and applications of video-language models:

\begin{itemize}
\item \textbf{Video Captioning:} Automatically generating descriptive text for video content.
\item \textbf{Video Question Answering (Video QA):} Answering questions based on the content of a video.
\item \textbf{Text-to-Video Generation:} Creating video sequences from textual descriptions.
\item \textbf{Video Retrieval:} Finding relevant videos based on textual queries or descriptions.
\item \textbf{Action Recognition:} Identifying and classifying actions depicted in video clips.
\end{itemize}

\noindent These models typically combine techniques from computer vision (for processing videos) and natural language processing (for handling text). Here, there are some Architecture and Models:

\begin{itemize}
\item \textbf{Encoder-Decoder Frameworks:} Commonly used where the encoder processes video frames to create a rich representation and the decoder generates the corresponding textual description.
\item \textbf{Transformer-based Models:} These models leverage attention mechanisms to handle the complexity of video and language data, such as the Vision Transformer (ViT) and variants like VideoBERT.
\item \textbf{Fusion Techniques:} Methods to effectively combine and align visual and textual data, such as cross-modal attention and joint embedding spaces.
\end{itemize}